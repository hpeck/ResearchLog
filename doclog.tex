\documentclass[12pt]{article}
\usepackage{graphicx}
\usepackage{makeidx}
\textwidth 6.5in
\oddsidemargin 0in
\evensidemargin 0in
\textheight 9in
\topmargin -0.5in
\usepackage{amsmath, amsfonts, amssymb, epsfig, epic, eepic, epsf}
\usepackage{subfigure}
\usepackage{float}
\usepackage{chemarr}
\usepackage[all]{xy}
\setcounter{secnumdepth}{5}
\setcounter{tocdepth}{4}
\usepackage{amsthm, setspace, stackrel, longtable}
\makeatletter
%%%%%%%%%%%%%%%%%%%%%%%%%%%%%%%%%%%%%%%%%%%%%%%%%%%%%%%%%%%%%%%%%%%%%

\begin{document}
\sffamily
\title{Log of Research Documents}
\date{\today}
\author{Hailee Peck}
\maketitle

\begin{enumerate}
\item Tumaneng, et al.: YAP mediates crosstalk between the Hippo and PI3K-TOR pathways by suppressing PTEN via miR-29.
\begin{itemize}
	\item YAP is main downstream target of mammalian Hippo pathway, 
	promotes organ growth
	\item YAP activates mTOR (major regulator of cell growth)
	\item YAP is phosphorylated and inhibited by LATS
	\item Cell density is known to regulate YAP phosphorylation and 
	activity
	\item Yorkie (Drosophila) induces expression of bantam microRNA that serves as a critical mediator of Yorkie's biological functions
	\item Establishes a functional link between Hippo and TOR in mammals

\end{itemize}

\item Kockel, et al.: Dynamic Switch of Negative Feedback Regulation in \textit{Drosophila} Akt-TOR Signaling
\begin{itemize}
	\item Three basic concepts of downregulating signaling pathways: control via specific inhibitory ligands/receptors, negative cross-regulation by distinct signaling pathways, auto-regulation by negative feedback mechanisms
	\item TORC1 and TORC2 both participate in Akt-TOR signaling but act at different levels in the pathway and integrate distinct stimuli: TORC2 responds to growth factors and might determine substrate specificity of Akt, TORC1 mediates signaling by amino acids and cellular energy stress.
	\item dAkt-TOR pathway in Drosophila regulates cell proliferation, developmental timing and sizing of cells, organs, and whole fly
	\item Drosophila has only one dAkt gene (mammals have 3)
	\item Phosphorylation of dAkt is regulated by negative feedback from Tsc1/Tsc2-TOR-S6K but independent of FoxO
	\item Negative feedback regulating dAkt activity is independent of S6K under normal TORC1 activity or dependent on S6K when TORC1 activity is high. Thus, S6K is a sensor of TORC1 that provides additional suppression of the signal when TORC1 is highly active. 
\end{itemize}

\item Zhao, et al.: The Hippo pathway in organ size control, tissue regeneration, and stem cell self-renewal
\begin{itemize}
	\item Hippo pathway limits organ size by phosphorylating and inhibiting Yki, a key regulator of proliferation and apoptosis
	\item Hippo pathway is regulated by cell polarity, cell adhesion and cell junction proteins
	\item Core components of Hippo pathway: warts (wts), hippo (hpo), salvador (sav), all tumor-suppressor genes
	\item Wts directly phosphorylates and inhibits Yki
	\item Merlin (Mer) and Expanded (Ex) were found to activate the Hippo pathway
	\item Fat protocadherin, a cell surface molecule, is an upstream regulator of Hippo pathway. Its activity is regulated by binding to Dachsous (Ds) and is modulated by Dco, Fj, and Lft. Dpp and Wg affect expression of Fj and Ds. 
	\item Yki induces cycE and E2F1 (regulation of cell proliferation), EGFR ligands and Jak-Stat ligands
	\item An imbalance of Hippo pathway activity in neighboring cells may induce cell competition through differential expression of dMyc in Drosophila
	\item Mechanism by which upstream regulators of the Hippo pathway are integrated to initiate or terminate signaling is not yet fully understood
\end{itemize}

\item Sun, Irvine: Cellular Organization and Cytoskeletal Regulation of the Hippo Signaling Network
\begin{itemize}
	\item Cell-cell junctions serve as platforms for Hippo signaling by localizing scaffolding proteins that interact with core components of the pathway
	\item Hippo was first discovered in Drosophila through the identification and characterization of genes that when mutated cause severe overgrowth phenotypes
	\item Hippo signaling is influenced by or crosstalks with multiple pathways that respond to growth factors, that promote growth linked to positional information, or that influence growth in response to nutritional and metabolic status. 
	\item Hippo signaling is also affected by contacts with neighboring cells and the ECm, and by mechanical forces. 
	\item Yki is downregulated by phosphorylation by the kinase Wts which promotes cytoplasmic localization of Yki (doesn't allow it into the nucleus)
	\item Core of the Hippo network: four proteins that regulate Yki, Hpo, Wts, Sav, and Mats
	\item Dachs influences Wts prtein levels and inhibits Wts association with Mats
	\item Mer promotes Wts activation by bringing Wts and Hpo together at cell membranes. 
	\item cell-cell junctions are under tension in Drosophila epithelia, promoting Yki activity
	\item Outstanding questions: Are different core components of the Hippo signaling network regulated by different upstream inputs?; How are different cytoskeleton dependent form sof regulation integrated and coordinated?; What additional cellular sites of Hippo and Warts activation remain to be discovered?
\end{itemize}

\end{enumerate}
\end{document}

