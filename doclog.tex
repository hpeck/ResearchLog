\documentclass[12pt]{article}
\usepackage{graphicx}
\usepackage{makeidx}
\textwidth 6.5in
\oddsidemargin 0in
\evensidemargin 0in
\textheight 9in
\topmargin -0.5in
\usepackage{amsmath, amsfonts, amssymb, epsfig, epic, eepic, epsf}
\usepackage{subfigure}
\usepackage{float}
\usepackage{chemarr}
\usepackage[all]{xy}
\setcounter{secnumdepth}{5}
\setcounter{tocdepth}{4}
\usepackage{amsthm, setspace, stackrel, longtable}
\makeatletter
%%%%%%%%%%%%%%%%%%%%%%%%%%%%%%%%%%%%%%%%%%%%%%%%%%%%%%%%%%%%%%%%%%%%%

\begin{document}
\sffamily
\title{Log of Research Documents}
\date{\today}
\author{Hailee Peck}
\maketitle

\begin{enumerate}
\item Tumaneng, et al.: YAP mediates crosstalk between the Hippo and PI3K-TOR pathways by suppressing PTEN via miR-29.
\begin{itemize}
	\item YAP is main downstream target of mammalian Hippo pathway, 
	promotes organ growth
	\item YAP activates mTOR (major regulator of cell growth)
	\item YAP is phosphorylated and inhibited by LATS
	\item Cell density is known to regulate YAP phosphorylation and 
	activity
	\item Yorkie (Drosophila) induces expression of bantam microRNA that serves as a critical mediator of Yorkie's biological functions
	\item Establishes a functional link between Hippo and TOR in mammals

\end{itemize}

\item Kockel, et al.: Dynamic Switch of Negative Feedback Regulation in \textit{Drosophila} Akt-TOR Signaling
\begin{itemize}
	\item Three basic concepts of downregulating signaling pathways: control via specific inhibitory ligands/receptors, negative cross-regulation by distinct signaling pathways, auto-regulation by negative feedback mechanisms
	\item TORC1 and TORC2 both participate in Akt-TOR signaling but act at different levels in the pathway and integrate distinct stimuli: TORC2 responds to growth factors and might determine substrate specificity of Akt, TORC1 mediates signaling by amino acids and cellular energy stress.
	\item dAkt-TOR pathway in Drosophila regulates cell proliferation, developmental timing and sizing of cells, organs, and whole fly
	\item Drosophila has only one dAkt gene (mammals have 3)
	\item Phosphorylation of dAkt is regulated by negative feedback from Tsc1/Tsc2-TOR-S6K but independent of FoxO
	\item Negative feedback regulating dAkt activity is independent of S6K under normal TORC1 activity or dependent on S6K when TORC1 activity is high. Thus, S6K is a sensor of TORC1 that provides additional suppression of the signal when TORC1 is highly active. 
\end{itemize}

\item Zhao, et al.: The Hippo pathway in organ size control, tissue regeneration, and stem cell self-renewal
\begin{itemize}
	\item Hippo pathway limits organ size by phosphorylating and inhibiting Yki, a key regulator of proliferation and apoptosis
	\item Hippo pathway is regulated by cell polarity, cell adhesion and cell junction proteins
	\item Core components of Hippo pathway: warts (wts), hippo (hpo), salvador (sav), all tumor-suppressor genes
	\item Wts directly phosphorylates and inhibits Yki
	\item Merlin (Mer) and Expanded (Ex) were found to activate the Hippo pathway
	\item Fat protocadherin, a cell surface molecule, is an upstream regulator of Hippo pathway. Its activity is regulated by binding to Dachsous (Ds) and is modulated by Dco, Fj, and Lft. Dpp and Wg affect expression of Fj and Ds. 
	\item Yki induces cycE and E2F1 (regulation of cell proliferation), EGFR ligands and Jak-Stat ligands
	\item An imbalance of Hippo pathway activity in neighboring cells may induce cell competition through differential expression of dMyc in Drosophila
	\item Mechanism by which upstream regulators of the Hippo pathway are integrated to initiate or terminate signaling is not yet fully understood
\end{itemize}

\item Sun, Irvine: Cellular Organization and Cytoskeletal Regulation of the Hippo Signaling Network
\begin{itemize}
	\item Cell-cell junctions serve as platforms for Hippo signaling by localizing scaffolding proteins that interact with core components of the pathway
	\item Hippo was first discovered in Drosophila through the identification and characterization of genes that when mutated cause severe overgrowth phenotypes
	\item Hippo signaling is influenced by or crosstalks with multiple pathways that respond to growth factors, that promote growth linked to positional information, or that influence growth in response to nutritional and metabolic status. 
	\item Hippo signaling is also affected by contacts with neighboring cells and the ECm, and by mechanical forces. 
	\item Yki is downregulated by phosphorylation by the kinase Wts which promotes cytoplasmic localization of Yki (doesn't allow it into the nucleus)
	\item Core of the Hippo network: four proteins that regulate Yki, Hpo, Wts, Sav, and Mats
	\item Dachs influences Wts prtein levels and inhibits Wts association with Mats
	\item Mer promotes Wts activation by bringing Wts and Hpo together at cell membranes. 
	\item cell-cell junctions are under tension in Drosophila epithelia, promoting Yki activity
	\item Outstanding questions: Are different core components of the Hippo signaling network regulated by different upstream inputs?; How are different cytoskeleton dependent form sof regulation integrated and coordinated?; What additional cellular sites of Hippo and Warts activation remain to be discovered?
\end{itemize}

\item Vidal, Cagan: Drosophila models for cancer research
\begin{itemize}
	\item The excess proliferation observed in cancer can be attributed to both a deregulation of the mechanisms underlying cell growth and a loss of inducers of apoptosis
	\item Yorkie is normally inactivated by Warts and mediates the transcription of cyclin E and DIAP1 (Drosophila inhibitor of apoptosis 1)
	\item Border cells migrate as an epithelial patch without undergoing an EMT. 
	\item In a homotypic environment where all cells in the tissue are scrib negative, cells display neoplastic growth and produce tumors
	\item In discrete clonal patches where scrib negative cells are mixed with wild type ones, the scrib negative cells undergo apoptotic death

\end{itemize}

\item Hemalatha, Prabhakara, Mayor: Endocytosis of Wingless via a dynamin-independent pathway is necessary for signaling in Drosophila wing discs
\begin{itemize}
	\item Wingless forms a spatial gradient across the boundary and activates distint concentration-dependent transcriptional programs ensuig coordinated tissue growth
	\item Study provides evidence for a mechanism wherein cells leverage multiple endocytic pathways to coordinate signaling during patterning
	\item Honestly not sure what the importance of this paper is in the scheme of looking at the Drosophila Hippo pathway
\end{itemize}

\item Shimobayashi, Hall: Making new contacts: the mTOR network in metabolism and signalling crosstalk
\begin{itemize}
	\item TOR pathway integrates stimuli from growth factors, nutrients, and cellular energy status to activate the metabolic pathways that ultimately drive cell growth
	\item Gives an overview of downstream effectors and upstream regulators of mTOR signaling, but (obviously) focused on mammalian
\end{itemize}

\item Harvey, Zhang, Thomas: The Hippo pathway and human cancer
\begin{itemize}
	\item Nice figure detailing the components of the Hippo pathway in Drosophila
	\item Else, mainly focuses on mammalian Hippo pathway
\end{itemize}

\item Lin, Othmer: A model for autonomous and non-autonomous effects of the Hippo pathway in Drosophila
\begin{itemize}
	\item Hippo pathway controls cell proliferation and apoptosis in Drosophila and mammalian cells
	\item Growth control in wing disc involves both local signals within the disc and system-wide signals such as insulin that coordinate growth across the organism
	\item pathways are tightly linked so strengths of interactions determine outcome -- Boolean model insufficient, need quantitative model
	\item Hippo pathway is a highly-conserved kinase cascade made of Hippo (Hpo), Warts (Wts), and adaptor proteins Salvador (Sav) and Mob as tumor suppressor (Mats)
	\item Key effector is Yorkie (Yki) and Wts is its master regulator
	\item Yki controlled by Wts through phosphorylation--once Yki is phosphorylated, it cannot enter the nucleus, so it cannot control cell proliferation or expression of genes upstream of Hippo module
	\item Hippo has two upstream modules: one based on Crumbs, Expanded, Merlin, and Kibra which affect Hippo kinase, and one based on Fat (Ft) and Dachsous (Ds) that regulate Wts. Input to Hippo assumed to be constant, so more important is second upstream module
	\item Ft and Ds are cadherins (calcium-dependent adhesion, bind cells together) which have intracellular, transmembrane, and extracellular domains. The ICDs mediate signaling within a cell while the ECD on adjacent cell membranes can associate to strengthen signaling and mediate cell-cell interaction
	\item Binding between Ft and Ds is regulated by Fj (Fj phosphorylates ECD of Ft, increasing its affinity to Ds, and phosphorylates ECD of Ds, decreasing its affinity to Ft)
	\item Signaling from ICD of Ft suppresses growth via Dachs (Dh): overexpression of Dh increases wing size, while Dh loss of function mutant decreases wing size
	\item Amount of Dh localized on membrane controls cell growth
	\item Effect of Ft on growth is not a strictly decreasing function of Ft level: overexpression of Ft above wild type levels decreases wing size and complete knockout of Ft increases wing size, but partial knockout of Ft decreases instead of increases wing size
	\item Effect of Ds is non-monotonic: loss of Ds enlarges wing discs but overexpression can either reduce or enhance growth 
	\item When Fj and Ds are co-overexpressed, the reduction in wing size is greater than for either separately
	\item \textbf{Central components for model: Ft, Ds, Dh, Riq, Wts, Yki}
	\item Phosphorylation: phosphate group (provided by ATP) added to a protein by kinases. THis alters the activity of a protein after the protein has already been formed. 
	\item Dh degraded more rapidly when bound to Ds so in the absence of Ds, Wts inhibition by Dh is increased since there is more of it around, leading to higher Yki bc less Wts to phosphorylate it (?)
	\item 
\end{itemize}

\item NIH proposal
\begin{itemize}
	\item Primary morphogens in the wing disc are wingless (Wg, a segment polarity gene), and decapentaplegic (Dpp, a bone morphogenic protein)
	\item \textbf{Goal: develop a multi-scale, 3D computational model of signaling in the wing disc}
	\item Essential to incorporate detailed structure in a horizontal slice of the disc, and transport and signaling in the vertical direction
	\item The morphogens Hh, Dpp, and Wg control the downstream network (they are each at the top of a signaling pathway)
	\item Local inputs to Hippo pathway are from links to adjacent cells via Ft and Ds, and the key effector is Yki, a cotranscription factor whose nuclear localization is controlled by the kinase Wts (when Yki phosphorylated by Wts, cannot enter nucleus)
	\item WAMND--what are my neighbors doing model
	\item Binding of Ds to Ft activates Wts by relieving the inhibition of Wts by Dachs. Activated Wts phosphorylated Yki to prevent entry into nucleus, controlling cell growth
	\item Incorporated in Hippo pathway: Ft, Ds, Fj, Dachs, Wts, Riq, Ex, and Yki
	\item Test whether feedback loops created by Yki control of Fj and Ds  expression leads to the observed spatial profiles of Ft and Ds (how do these loops affect Yki activity within a cell)

\end{itemize}

\item Zhang, Lai: Mob as tumor suppressor is regulated by bandtam microRNA through a feedback loop for tissue growth control
\begin{itemize}
	\item Yki upregulates transcription of Expanded and Merlin in Drosophila
	\item bantam microRNA is a downstream target of Hippo pathway; promotes tissue growth by stimulating cell proliferation and inhibiting cell apoptosis. 
	\item Show: ban indirectly regulates mats expression in developing tissues
	\item ban is sufficient to increase level of Mats expression, but does not regulate mats at the transcript level
	\item mats is the only core component of the Hpo pathway to be a potential target of ban miRNA. 
	\item ban miRNA functions to increase the level of mats expression through a feedback loop, but indirectly--not sure what the factor between ban and mats is
	\item Yki also mediates transcriptional regulation of several upstream genes such as ex and mer
\end{itemize}

\item Djiane, et al.: Notch Inhibits Yorkie Activity in Drosophila Wing Discs
\begin{itemize}
	\item Notch is required throughout the wing disc for growth but the effects of its overactivation are most noticeable at the edge of the wing disc
	\item Notch activation releases intracellular part of Micd receptor which enters nucleus, binds to transcription factor SuH, and turns on transcription of target genes
	\item Epithelial cells require Notch and depend on Sav/Wts/Hpo pathway to regulate growth 
	\item DNA binding transcription factors of Yki: Scalloped/TEAD, Homothorax, p53
	\item By modifying the ration between Sd, Vg, and Yki, Notch signaling prevents Yki from activating its targets in the wing pouch
	\item Notch could either act upstream of Yki by activating the SWH pathway or downstream of Yki by inhibiting Yki's transcriptional activity...results suggest Notch acts either at the level or downstream of Yki
	\item Vg mediates the repressive effects of Notch on Expanded expression
	\item Vg inhibits expression of two Yki targets (ex-lacZ, IAP2B2C-lacZ) in the wing pouch
	\item Notch activity can inhibit Yki under circumnstances where Yki acts together with Sd by promoting expressio of Vg, a cofactor for Sd, counteracting the effects of Yki
	\item 
\end{itemize}
\item Konsavage, Yochum: Intersection of Hippo/YAP and Wnt/$\beta$-catenin signaling pathways
\begin{itemize}
	\item Mutation in Wingless gene causes loss of wings in Drosophila (hence the name...)
	\item Wnt is the human homolog of Wg
	\item Wnt genes mediate short range paracrine signaling
	\item Loss of function mutations in Warts resulted in strong tissue overgrowth
	\item Loss of function mutations in each of Warts, Hippo, Salvador, and Mats proteins gave a similar phenotype so hypothesized that they functioned as part of a signaling pathway
	\item ``Yki acts as a transcriptional coactivator protein and it interacts with the transcription factor Scalloped to drive the expression of genes that promote cell growth and inhibit apoptosis"
	\item Fat senses cell-cell contacts by binding Dachsous
	\item Border cells in the fly imaginal disc express Wg and Vestigial (Vg), which is a transcription factor that specifies wing cells
	\item Border cells provide a feed forward signal that involes Wg which recruits non-wing cells from surrounding tissue and instructs them to become wing cells
	\item Scalloped/Yki directly activate Vg which results in the conversion of nonwing cells to wing cells
	\item E-cadherin expression is lost during the progression and metastasis of tumors (because tumors want to be able to grow even if they're not in contact with other cells or a substrate)
\end{itemize}

\item Alexandre, Baena-Lopez, Vincent: Patterning and growth control by membrane-tethered Wingless
\begin{itemize}
	\item Spread of Wingless is dispensable for patterning and growth even though it contributes to increasing cell proliferation
	\item Wg initially expressed throughout wing and then later only at a narrow stripe at the boundary between front and back of body
	\item Near this boundary, High activity activates senseless expression $\rightarrow$ wing margin fates. Further away, low signal stimulates vestigial, Distal-less and frizzled, which are more sensitive to signalling. 
	\item Reporters of Wg singalling: senseless, vestigial, Distal-less, frizzled 3
	\item Wild type Wg can spread and signal over a few cell diameters
	\item Wg knockout flies are morphologically normal but have a delay in entering fly puberty and are less fit (fail to eclose)
	\item  Wg is first expressed throughout wing precursor and eventually is restricted to a narrow stripe at the dorsoventral boundary, where it will spread to form a gradient
\end{itemize}

\item Attisano, Wrana: Signal integration in TGF-$\beta$, Wnt, and Hippo pathways
\begin{itemize}
	\item Smads are direct receptor substrates that upon phosphorylation accumulate int he ncelus to regulate transcription through interactings with DNA-binding partners
	\item Smad pathway is key for directing TGF-$\beta$ transcriptional responses
	\item Hippo pathway is activated through cell-cell contact or upon polarization of epithelial cells
	\item Smads homolog in Drosophila? Mad, Medea, Dad, Smox 
\end{itemize}
\end{enumerate}
\end{document}

